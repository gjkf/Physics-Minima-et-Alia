%!TEX ROOT=formularioFisica.tex

\section{Dimostrazioni}
Di seguito verrano inserite delle dimostrazioni di formule che si possono incontrare nel formulario.
La sezione non è assolutamente necessaria, è semplicemente per i più curiosi.

\subsection{Dinamica}
\subsubsection{Oscillatore armonico}
\begin{center}
  \begin{tikzpicture}
    \draw (0,0) -- (5,0);
    \draw[pattern=north east lines] (0,0) -- (1,0) -- (1,1) -- (0,1);
    \draw[decoration={coil},decorate] (1,0.5) -- (3,0.5);
    \draw (3.5,0.5) circle (0.5);
  \end{tikzpicture}
\end{center}
Quando questa massa si muove di una distanza $x_0$ e poi viene rilasciata, tenderà a tornare alla
posizione originale. La forza che agisce è pari a $F=-ka$. È una forza non costante, quindi
può essere vista come derivata
\begin{equation*}
  F = m x'' \rightarrow -kx = mx''
\end{equation*}
Questa ora è iventata un'equazione differenziale lineare del secondo ordine. Qundi si risolve
\begin{equation*}
  m\lambda^2+k=0 \rightarrow \lambda=\pm i \sqrt{\frac{k}{m}}
\end{equation*}
e quindi
\begin{equation*}
  x=c_1\cos\left(\sqrt{\frac{k}{m}}t\right)+c_2\sin\left(\sqrt{\frac{k}{m}}t\right)
\end{equation*}
Sapendo che $x(0)=0$, si sostituisce e ottiene $x_0=c_1$. Ora si deriva
\begin{equation*}
  x'=-c_1\sqrt{\frac{k}{m}}\sin\left(\sqrt{\frac{k}{m}}t\right)+
  c_2\sqrt{\frac{k}{m}}\cos\left(\sqrt{\frac{k}{m}}t\right)
\end{equation*}
Ora sapendo che $x'(0)=0$, sostiuiamo e otteniamo che $c_2=\sqrt{\frac{k}{m}}$. A questo punto
otteniamo la soluzione
\begin{equation*}
  x=x_0\cos\left(\sqrt{\frac{k}{m}}t\right)
\end{equation*}
Posto $\omega=\sqrt{\frac{k}{m}}$ si ottiene $x=x_0\cos\omega t$.

\subsubsection{Lavoro}
Si prenda un corpo di massa $m$ con $v_0=0$ su cui è applicata una forza $\vec{F}$ per farlo
accelerare a velocità $v$. Se si definisce $1$ l'istante iniziale e $2$ quello finale, nella
meccanica classica il lavoro è pari a
\begin{equation*}
  L = \int\limits_{1}^{2} \vec{F}\dif\vec{r}
\end{equation*}
Dato che
\begin{equation*}
  \vec{F} = \der{\vec{q}}{t}
\end{equation*}
si ha che
\begin{align*}
  L &= \int\limits_{1}^{2} \der{\vec{q}}{t}\dif\vec{r}=
  \int\limits_{1}^{2}\dif\vec{q}\der{\vec{r}}{t}=\int\limits_{1}^{2} \vec{v}\dif\vec{q}=
  \int\limits_{1}^{2} mv\dif\vec{v}=\\
  &=m \int\limits_{1}^{2} v\dif v=
  \left.m \frac{v^2}{2}\right|_0^v=\frac{1}{2}mv^2
\end{align*}

\subsection{Circuiti elettrici}
\subsubsection{Circuito RC}
\begin{center}
  \begin{tikzpicture}\ctikzset{bipoles/length=.9cm}
    \draw(1,0)
      to[battery1] (0,0)
      to[short] (0,1)
      to[C] (2,1)
      to[R] (2,0)
      to[switch] (1,0);
  \end{tikzpicture}
\end{center}
Per la legge di Kirchhoff
\begin{equation*}
  \mathcal{E} = \mathcal{E}_R + \mathcal{E}_C
\end{equation*}
e quindi
\begin{equation*}
  \mathcal{E} = Ri + \frac{q}{C}=R\der{q}{t}+\frac{q}{C}
\end{equation*}
Alla chiusura dell'interruttore sappiamo che $q(0)=0$. Quindi abbiamo un problema di Cauchy.
\begin{equation*}
  \begin{cases}
    \mathcal{E} = R\der{q}{t}+\frac{q}{C}\\
    q(0) = 0
  \end{cases}
\end{equation*}
L'equazione differenziale è a variabili separabili, quindi islando le funzioni in $t$
\begin{equation*}
  \der{q}{t}=\frac{C\mathcal{E}-q}{RC}
\end{equation*}
e successivamente
\begin{align*}
  -\int\frac{-\dif q}{C\mathcal{E}-q}&=\int\frac{1}{RC}\dif t\\
  \ln\abs{C\mathcal{E}-q}&=\frac{1}{RC}t+c\\
  \abs{C\mathcal{E}-q} &= e^{\frac{1}{RC}t}e^c\\
  C\mathcal{E}-q&=\pm e^{-\frac{t}{RC}}\underbrace{e^c}_{\omega}\\
  q&=-C-\omega e^{-\frac{t}{RC}}\\
  q(t) &= C\mathcal{E}\left(1-e^{-\frac{t}{RC}}\right)
\end{align*}
Avendo anche l'altra informazione, sostituiamo
\begin{equation*}
  0=C\mathcal{E}-\omega e^0 \rightarrow \omega = C\mathcal{E}
\end{equation*}

\subsubsection{Circuito RL}
\begin{center}
  \begin{tikzpicture}\ctikzset{bipoles/length=.9cm}
    \draw(0,0) 
      to[battery1] (1,0)
      to[switch] (2,0)
      to[R] (2,1)
      to[L] (1,1)
      to[short] (0,1)
      to[short] (0,0);
  \end{tikzpicture}
\end{center}
Per la legge di Kirchhoff
\begin{equation*}
  \mathcal{E} = L\der{i}{t}+Ri
\end{equation*}
Essendo un'equazione differenziale lineare del primo ordine in $i(t)$, è possibile riscriverla
come
\begin{equation*}
  \der{i}{t}=-\frac{R}{L}i+\frac{\mathcal{E}}{L}
\end{equation*}
La soluzione è perciò del tipo
\begin{equation*}
  i(t) = Ce^{-\frac{R}{L}}+\frac{V}{R}
\end{equation*}
Avendo però la condizione data dal problema di Cauchy $i(0)=0$, si ottiene
\begin{equation*}
  i(t)=\frac{\mathcal{E}}{R}\left( 1-e^{-\frac{R}{L}}t \right)
\end{equation*}

\subsection{Relatvità}
\subsubsection{Vento dell'etere}
L'esperimento del vento dell'etere può essere riassunto così
\begin{center}
  \begin{tikzpicture}[square/.style={regular polygon,regular polygon sides=4}]
    \coordinate (A) at (0,0);
    \coordinate (B) at (3,0);
    \coordinate (C) at (0,3);

    \node at (A) [square,draw](A){A};
    \node at (B) [square,draw](B){B};
    \node at (C) [square,draw](C){C};

    \draw[|<->|] (A) -- (B)
      node[pos=0.5,below]{$L$};
    \draw[|<->|] (A) -- (C)
      node[pos=0.5,left]{$L$};

    \foreach\y in {2,1.5,1}{%
      \draw[->] (3,\y) -- ++(-2,0);
    }
  \end{tikzpicture}
\end{center}
Definendo $v_M$ la velocità del moto e $v_V$ la velocità del vento, si può immaginare un corpo che
segue la rotta A-B e un altro che segue quella A-C. La teoria dice che i tempi impiegati dai due 
moti devono essere diversi se il vento dell'etere esistesse. Infatti si ha che
\begin{align*}
  t_\perp&<t_\|\\
  \intertext{Considerato che la velocità d'andata e di ritorno sono diverse, si devono sommare, si
  hanno che la somma è pari a $\frac{L}{v_M-v_V} + \frac{L}{v_M+v_V}$}
  \frac{2L}{\sqrt{v_M^2-v_V^2}}&<\frac{2v_M^2}{v_M^2-v_V^2}\\
  \frac{1}{v_M^2-v_V^2}&<\frac{v_M^2}{{(v_M-v_V)}^2}\\
  1&<\frac{v_M^2(v_M^2-v_V^2)}{(v_M^2-v_V^2)}\\
  1&<\frac{v_M^2}{v_M^2-v_V^2}\\
  v_M^2-v_V^2&<v_M^2
\end{align*}

\subsubsection{Dilatazione del tempo}
\begin{align*}
  {\left( \frac{c\Delta t}{2} \right)}^2&=l^2+{\left( \frac{v\Delta t}{2} \right)}^2\\
  \frac{c^2}{4}{(\Delta t)}^2&=l^2+\frac{v^2{(\Delta t)}^2}{4}\\
  (c^2-v^2){(\Delta t)}^2&={(2l)}^2\\
  (c^2-v^){(\Delta t)}^2 &= c^2{(\Delta t')}^2\\
  {(\Delta t)}^2&= \frac{c^2{(\Delta t')}^2}{c^2-v^2}\\
  \Delta t &= \frac{\Delta t'}{\sqrt{1-\frac{v^2}{c^2}}}
\end{align*}

\subsubsection{Invariante spazio-temporale}
\begin{equation*}
  {(\Delta\sigma)}^2={(c\Delta t)}^2-{(\Delta S)}^2
\end{equation*}
Dove $\Delta S$ è la differenza della variazione di tutte le coordinate. Si ha quindi che
\begin{align*}
  \Delta x' &= \gamma(\Delta x-v\Delta t)\\
  \Delta y' &= \Delta y\\
  \Delta z' &= \Delta z\\
  \Delta t' &= \gamma \left( \Delta t-\frac{v}{c^2}\Delta x \right)
\end{align*}
Sostituendo
\begin{align*}
  {(\Delta\sigma')}^2 
  &= {(c\Delta t')}^2-{(\Delta x')}^2-{(\Delta y')}^2-{(\Delta z')}^2\\
  &= {\left(c\left(\gamma \left( \Delta t-\frac{v}{c^2}\Delta x \right)\right)\right)}^2-
  {\left( \gamma \left( \Delta x-v\Delta t \right) \right)}^2-\\
  &{(\Delta y)}^2-{(\Delta z)}^2\\
  &= c^2\gamma^2 \left[ {(\Delta t)}^2+\frac{v^2}{c^2}{(\Delta x)}^2-\frac{2v}{c^4}{(\Delta x)}^2 -
\frac{2v}{c^2}\Delta x\Delta t\right]-\\
&\gamma^2({(\Delta x)}^2+v^2{(\Delta t)}^2-2\Delta x\Delta t)-{(\Delta y)}^2-{(\Delta z)}^2\\
&=\gamma^2 \left[ c^2{(\Delta t)}^2+\frac{v^2}{c^4}{(\Delta x)}^2-v{(\Delta t)}^2 \right]-
{(\Delta y)}^2-{(\Delta z)}^2\\
&=\gamma^2 \left[ (c^2-v^2){(\Delta t)}^2-\left( 1-\frac{v^2}{c^2}
  {(\Delta x)}^2 \right){(\Delta x)}^2 \right]-\\
  &{(\Delta y)}^2-
  {(\Delta z)}^2\\
  &=\frac{c^2}{c^2v^2}\left[ (c^2-v^2){(\Delta t)}^2-\left( \frac{c^2-v^2}{v^2}
  \right){(\Delta x)}^2 \right]-\\
  &{(\Delta y)}^2-{(\Delta z)}^2\\
  &=c^2{(\Delta t)}^2-{(\Delta x)}^2-{(\Delta y)}^-{(\Delta z)}^2\\
  &={(\Delta\sigma)}^2
\end{align*}

\subsubsection{Lavoro relativistico}
Si prenda una particella di massa $m$ con $v_0=0$ su cui è applicata una forza $\vec{F}$ per farla
accelerare a velocità $v$. Se si definisce $1$ l'istante iniziale e $2$ quello finale, nella
meccanica classica il lavoro è pari a
\begin{equation*}
  L = \int\limits_{1}^{2} \vec{F}\dif\vec{r}
\end{equation*}
Dato che
\begin{equation*}
  \vec{F} = \der{\vec{q}}{t}
\end{equation*}
si ha che
\begin{align*}
  L &= \int\limits_{1}^{2} \der{\vec{q}}{t}\dif\vec{r}= \int\limits_{1}^{2}\vec{v}\dif\vec{q}=
  \left. \vphantom{\int}vq\right|_0^q-\int\limits_{0}^{v} q\dif v=\\
  &=vq-\int\limits_{0}^{v} m\gamma v\dif v=  
  vq-m \int\limits_{0}^{v} \frac{v\dif v}{\sqrt{1-\frac{v^2}{c^2}}}=\\
  &=vq-m \int\limits_{0}^{v} \left( 1-\frac{v^2}{c^2} \right)^{-\frac{1}{2}}\dif v=
  vq-\frac{mc^2}{2}
  \int\limits_{0}^{v} -\frac{2v}{c^2}\left( 1-\frac{v^2}{c^2} \right)^{-\frac{1}{2}}\dif v=\\
  &=vq+\frac{mc^2}{2}\left.\left( 1-\frac{v^2}{c^2}  \right)^{\frac{1}{2}}\cdot2\right|_0^v=
  vq+\frac{mc^2}{2} \left[ \frac{\left(1-\frac{v^2}{c^2}\right)^{\frac{1}{2}}}{\frac{1}{2}}-
  \frac{1}{\frac{1}{2}}\right]=\\
  &=vq+mc^2\left(\sqrt{1-\frac{v^2}{c^2}}-1\right)=\frac{mv}{\sqrt{1-\frac{v^2}{c^2}}}\cdot v+
  mc^2\sqrt{1-\frac{v^2}{c^2}}-mc^2=\\
  &=\frac{mv^2+mc^2\left( 1-\frac{v^2}{c^2} \right)}{\sqrt{1-\frac{v^2}{c^2}}}-mc^2=
  c^2\gamma-mc^2
\end{align*}

\subsubsection{Rapporto Quantità di moto-energia}
Si sa che $q=\gamma mv$ e $E=\gamma mc^2$.
\begin{align*}
  q&=\gamma mv\\
  cq&=\gamma mvc\\
  c^2q^2=\gamma^2m^2v^2c^2&\quad E^2=\gamma^2m^2c^4\\
  E^2-cq^2&=m^2\gamma^2c^2(c^2-v^2)\\
  E^2-cq^2&=m\frac{c^2}{c^2-v^2}c^2(c^2-v^2)\\
  E^2-c^2q^2&=m^2c^4
\end{align*}
