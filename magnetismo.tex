%!TEX ROOT=formularioFisica.tex

\section{Magnetismo}
Il magnetismo si occupa di come i magneti interagiscano fra di loro e con i corpi circostanti.Il 
campo magnetico ($\vec{B}$) � quello che influenza i corpi al suo interno. La sua direzione �
sempre dal polo NORD a quello SUD.\\ 
La sua unit� di misura � il \emph{Tesla} ($T = \frac{N}{Am}$)\\
Pu� essere definito in due modi, prenderemo in considerazione quello pi� usato in ambito liceale.

\subsection{Forza in un campo magnetico}
\begin{equation*}	
\vec{F} = i\cdot\vec{l}\times\vec{B}
\end{equation*}
Si ricordi che $\times$ indica il \hyperref[subsec:vettori:prodottoVettoriale]{prodotto vettoriale} 
tra due vettori. $\vec{l}$ � un vettore che rappresenta un filo che attraversa il campo magnetico.
$i$: corrente

\subsection{Legge di Biot-Savant}
La legge di Biot-Savant definisce un campo magnetico generato da un filo rettilineo percorso da
una corrente.\\
La direzione � la tangente alla linea di forza generata dal filo, il verso � antiorario se la
corrente esce, orario altrimenti.
\begin{equation*}
B = \frac{\mu_0}{2\pi}\frac{i}{d}
\end{equation*}
\hyperref[tab:mu0]{$\mu_0$}: $4\pi\cdot10^{-7}\,\text{N/A}^2$\\
$i$: corrente\\
$d$: distanza dal filo\\

\subsection{Legge di Amp�re}
La legge di Amp�re descrive la forza che intercorre tra due fili paralleli attraversati da una
corrente.
\begin{equation*}
F = \frac{\mu_0}{2\pi}\frac{i_1i_2}{d}l
\end{equation*}
\hyperref[tab:mu0]{$\mu_0$}: $4\pi\cdot10^{-7}\,\text{N/A}^2$\\
$i$: corrente\\
$d$: distanza tra i due fili\\
\\
Molto spesso si vuole definire la forza su di una certa dimensione del filo, quindi si ottiene
la formula in questa forma:
\begin{equation*}
\frac{F}{l} = \frac{\mu_0}{2\pi}\frac{i_1i_2}{d}l
\end{equation*}
La formula calcola il modulo del vettore. La forza � attrattiva se i versi delle correnti sono 
uguali, repulsiva altrimenti.

\subsection{Forza di Lorenz}
La forza di Lorenz � la forza che una carica risente all'interno di un campo magnetico.
\begin{equation*}
\vec{F} = q\vec{v}\times\vec{B}
\end{equation*}

Si presti attenzione al verso di questa forza. Se la carica � \emph{positiva} si usi la solita
regola della mano, altrimenti si inverta semplicemente il verso.\\\\	
Una particolarit� � che una carica che entra in un campo magnetico viene deviata dal suo percorso.
Il moto che ne deriva � \emph{elicoidale}. Ovviamente questo nel caso pi� generale, il disegno
di seguito aiuter� a capire in che modo l'elettrone viene spostato

\begin{center} % https://tex.stackexchange.com/questions/129860/helix-on-a-cylinder
	\tdplotsetmaincoords{70}{15}
	\begin{tikzpicture}[tdplot_main_coords]
	
		% --- Independent parameters ---
		\def\h{3}                          % cylinder height
		\pgfmathtruncatemacro\tA{350}      % A angle
		\def\zA{1}                         % A applicate
		\pgfmathtruncatemacro\tB{150}      % B angle
		\def\zB{2}                         % B applicate
		\pgfmathtruncatemacro\n{3}         % number of additional turns
		\pgfmathtruncatemacro\NbPt{100}    % number of dots for drawing the helix portion
		\def\rhelixdots{0.05}              % radius of dots forming helix
		\def\rAB{0.05}                     % radius of A and B dots
		
		% lower circle
		\draw[black,very thin] (1,0,0) 
		\foreach \t in {2,3,...,360}
		{
			--({cos(\t)},{sin(\t)},0)
		}
		--cycle;
		
		% upper circle
		\draw[black,very thin] (1,0,\h) 
		\foreach \t in {2,4,...,360}
		{
			--({cos(\t)},{sin(\t)},\h)
		}
		--cycle;
		
		% --- Draw helix ---
		\pgfmathsetmacro\tone{\tA}
		\pgfmathsetmacro\tlast{\tB+\n*360}
		\pgfmathsetmacro\ttwo{\tone+(\tlast-\tone)/(\NbPt-1)}
		\pgfmathsetmacro\p{360*(\zB-\zA)/(\tB-\tA+360*\n)}
		\foreach \t in {\tone,\ttwo,...,\tlast}{%
			\fill[red!80] ({cos(\t)},{sin(\t)},{\p*(\t-\tA)/360+\zA}) circle[radius=\rhelixdots];
		}
		
		% --- Draw A and B ---
		\fill[blue] ({cos(\tA)},{sin(\tA)},\zA) circle [radius=\rAB]node[right]{$A$};
		\fill[blue] ({cos(\tB)},{sin(\tB)},\zB) circle [radius=\rAB]node[left]{$B$};
	\end{tikzpicture}
\end{center}
Questa � una rappresentazione del moto di una cairca all'interno del campo magnetico. Alcune formule
permettono di trovare il \emph{raggio}, \emph{periodo} e il \emph{passo} dell'elica.
\begin{alignat*}{2}
r &= \frac{m\cdot v}{q\cdot B} &\qquad  T &= \frac{m\cdot v_x}{q\cdot B}\\
\delta S &= v_y = T &\qquad F &= m\frac{v^2}{r}
\end{alignat*}
L'ultima formula di queste quattro trova la forza centripeta che mantiene la carica all'interno del
moto. Questa � la forza di Lorenz (supposto che $\vec{B}\perp\vec{v}$).\\
Sempre supposto che $\vec{B}\perp\vec{v}$, il periodo si pu� scrivere anche come
\begin{equation*}
T = \frac{2\pi\cdot m}{q\cdot B}
\end{equation*}

\subsection{Selettore di velocit�}
Il selettore di velocit� � un particolare dispositivo che permette di "selezionare" alcune cariche che
vanno solo ad una determinata velocit� attraverso una fessura. Il loro utilizzo � molto ampio, 
specialmente in dispositivi come televisioni a tubo catodico.\\
Il loro funzionamento si basa su due campi $\vec{E}$ e $\vec{B}$ uniformi incrociati fra di loro.\\
Sulla carica deve agire una forza $\vec{F}$ pari a
\begin{equation*}
\vec{F} = q\vec{E} + q\vec{v}\times\vec{B}
\end{equation*}
Una carica passa nel selettore se e solo se la sua velocit� � pari a
\begin{equation*}
v = \frac{E}{B}
\end{equation*}

\subsection{Conduttori nei metalli}
In questa sottosezione si trovano le formule per trovare la carica massima che pu� ottenere e la
velocit� di deriva degli elettroni.\\

\subsubsection{Carica massima}
\begin{equation*}
i = n\cdot A\cdot v_d\cdot e
\end{equation*}
\hyperref[tab:e-]{$e$}\\
$n$: numero di elettroni di conduzione77
$A$: sezione del conduttore

\subsubsection{Velocit� di deriva}
\begin{equation*}
v_d = \frac{i}{n\cdot A\cdot e}
\end{equation*}
\hyperref[tab:e-]{$e$}\\
$n$: numero di elettroni di conduzione77
$A$: sezione del conduttore