%!TEX ROOT=formularioFisica.tex

\section{Circuiti elettrici}\label{sec:circElettr}
Questa sezione � strettamente legata alla precedente ma, vista l'importanza ne merita una sua.\\
I circuiti elettrici sono usati ovunque nel mondo d'oggi e hanno una simbologia tutta loro. Di 
seguito viene riportato un circuito elementare.

\begin{center}
	� � \begin{circuitikz}
			\draw (0,0)
			to[battery1, l=9<\volt>] (0,2)
			to[short] (2,2)
			to[R=$R_1$, l=10<\kilo\ohm>] (2,0)
			to[short] (0,0);
		\end{circuitikz}
\end{center}

Questo circuito non ha una funzione in quanto � composto solo da una batteria
\begin{center}
	\begin{circuitikz}
		\draw (0,0) to[battery1, l=9<\volt>] (2,0);
	\end{circuitikz}
\end{center}
ed una resistenza
\begin{center}
	\begin{circuitikz}
		\draw (0,0) to[R=$R_1$, l=10<\kilo\ohm>](2,0);
	\end{circuitikz}
\end{center}

L'ultimo simbolo che verr� frequentemento usato � quello del condensatore
\begin{center}
	\begin{circuitikz}
		\draw (0,0) to[C=$C_1$, l=10<\pico\farad>](2,0);
	\end{circuitikz}
\end{center}

Ogni tanto si pu� anche trovare un interruttore che pu� interrompere il flusso di carica
\begin{center}
	\begin{circuitikz}
		\draw (0,0) to[switch](2,0);
	\end{circuitikz}
\end{center}

Per gli esercizi, si vada \hyperref[ex:circElettr]{qui}.

\subsection{Corrente elettrica}
Indica quante cariche (di solito elettroni) si muovono all'interno di un corpo conduttore durante un 
tempo determinato.

\subsection{Collegamenti}
Ci sono 2 tipi di collegamenti: \textbf{in serie} ed \textbf{in parallelo}.\\
Se due componenti sono in serie, la stessa corrente $i$ li attraversa e sono posti uno dietro 
l'altro. Se sono in parallelo invece la corrente si divide.

\subsubsection{In serie}
\begin{center}
	\begin{circuitikz}
		\draw (-1, 0)
		to [short] (0, 0)
		to [R] (1,0)
		to [short] (2,0)
		to [R] (3,0)
		to [short] (4,0);
	\end{circuitikz}
\end{center}

\subsubsection{In parallelo}
\begin{center}
	\begin{circuitikz}
		\draw 
		(-1,0)
		to [short, o-] (0,0)
		to [short, *-] (0,1)
		to [short] (1,1)
		to [R] (2,1)
		to [short] (3,1)
		to [short] (3,0)
		
		(0,0) 
		to [short, *-] (0,-1)
		to [short] (1,-1)
		to [R] (2,-1)
		to [short] (3,-1)
		to [short] (3,0)
		
		(3,0)
		to [short, *-o] (4,0);
	\end{circuitikz}
\end{center}

\subsection{Condensatori}
Un condensatore � composto da due \emph{armature} che hanno cariche di segno opposto, una positiva e
l'altra negativa. Il compito di un condensatore all'interno di un circuito � di immagazzinare carica
e rilasciarla. Viene usato ad esempio nei flash delle macchine fotografiche dove viene
rilasciata in un attimo la carica immagazzinata in un certo tempo.\\
La loro unit� di misura � il Farad (F).

\subsubsection{Capacit� equivalente di condensatori in \emph{serie}}
\begin{center}
	\begin{circuitikz}
		\draw
		(0,0)
		to [short] (1,0)
		to [C=$C_1$] (2,0)
		to [short] (3,0)
		to [C=$C_2$] (4,0)
		to [short] (5,0);
	\end{circuitikz}
\end{center}

Questo "circuito" pu� essere ridotto ad un solo condensatore la cui capacit� deve essere
\begin{equation*}
\frac{1}{C_e} = \frac{1}{C_1} + \frac{1}{C_2}
\end{equation*}

\subsubsection{Capacit� equivalente di condensatori in \emph{parallelo}}
\begin{center}
	\begin{circuitikz}
		\draw 
		(-1,0)
		to [short, o-] (0,0)
		to [short, *-] (0,1)
		to [short] (1,1)
		to [C=$C_1$] (2,1)
		to [short] (3,1)
		to [short] (3,0)
		
		(0,0) 
		to [short, *-] (0,-1)
		to [short] (1,-1)
		to [C=$C_2$] (2,-1)
		to [short] (3,-1)
		to [short] (3,0)
		
		(3,0)
		to [short, *-o] (4,0);
	\end{circuitikz}
\end{center}

Questo "circuito" pu� essere ridotto ad un solo condensatore la cui capacit� deve essere
\begin{equation*}
C_e = C_1 + C_2
\end{equation*}

\subsubsection{Propriet� di condensatori in serie ed in parallelo}
Una caratteristca di questi due collegamenti � quello che rimane costante tra i
condensatori.\\[\baselineskip]
Due condensatori \textbf{in serie} mantengono la stessa $Q$.\\
Due condensatori \textbf{in parallelo} mantengono la stessa ddp ($\Delta V$).

\subsection{Resistenze}
Le resistenze servono appunto a opporre resistenza alla corrente elettrica. Permettono di collegare
componenti con una tensione di rendimento bassa a circuiti che ne forniscono una alta.\\
La loro unit� di misura � l'Ohm ($\Omega$).

\subsubsection{Leggi di Ohm}
Le leggi di Ohm mettono in relazione ddp e resistenza, e le caratteristiche fisiche di una 
resistenza.

\begin{alignat*}{2}
R &= \frac{\Delta V}{i} &\qquad R &= \rho\frac{l}{S}
\end{alignat*}
La prima mette in relazione la ddp e la corrente con la resistenza.\\
La seconda mette in relazione la resistivit� ($\rho$), la lunghezza del corpo ($l$) e la sezione 
($S$) con la resistenza.

\subsubsection{Capacit� equivalente di resistenze in \emph{serie}}
\begin{center}
	\begin{circuitikz}
		\draw
		(0,0)
		to [short] (1,0)
		to [R=$R_1$] (2,0)
		to [short] (3,0)
		to [R=$R_2$] (4,0)
		to [short] (5,0);
	\end{circuitikz}
\end{center}

Questo "circuito" pu� essere ridotto ad un solo condensatore la cui capacit� deve essere
\begin{equation*}
R_e = R_1 + R_2
\end{equation*}

\subsubsection{Capacit� equivalente di resistenze in \emph{parallelo}}
\begin{center}
	\begin{circuitikz}
		\draw 
		(-1,0)
		to [short, o-] (0,0)
		to [short, *-] (0,1)
		to [short] (1,1)
		to [R=$R_1$] (2,1)
		to [short] (3,1)
		to [short] (3,0)
		
		(0,0) 
		to [short, *-] (0,-1)
		to [short] (1,-1)
		to [R=$R_2$] (2,-1)
		to [short] (3,-1)
		to [short] (3,0)
		
		(3,0)
		to [short, *-o] (4,0);
	\end{circuitikz}
\end{center}

Questo "circuito" pu� essere ridotto ad una sola resistenza la cui capacit� deve essere
\begin{equation*}
\frac{1}{R_e} = \frac{1}{R_1} + \frac{1}{R_2}
\end{equation*}
Vige anche questa propriet�:
\begin{equation*}
R_1i_1 = R_2i_2
\end{equation*}

\subsubsection{Propriet� di condensatori in serie ed in parallelo}
Una caratteristca di questi due collegamenti � quello che rimane costante tra le 
resistenze.\\[\baselineskip]
Due resistenze \textbf{in serie} mantengono la stessa $i$.\\
Due resistenze \textbf{in parallelo} mantengono la stessa ddp ($\Delta V$).

\subsection{Tabella riassuntiva delle formule}

\begin{center}
	\begin{tabular}{c | c | c}
		& Serie & Parallelo \\ \hline
		Condensatori & $\frac{1}{C_e} = \frac{1}{C_1} + \frac{1}{C_2}$ & $C_e = C_1 + C_2$\\ \hline
		Resistenze & $R_e = R_1 + R_2$ & $\frac{1}{R_e} = \frac{1}{R_1} + \frac{1}{R_2}$\\
	\end{tabular}
\end{center}

\subsection{Generatori}
I generatori generano, abbastanza logicamente, corrente elettrica. Ce ne sono di due tipi:
\emph{Ideali} e \emph{Reali}. L'unica differenza � che i generatori reali hanno resistenza
interna.

\subsubsection{Forza elettromotrice}
\begin{equation*}
\mathcal{E} = \overbrace{iR}^{\Delta V} + \underbrace{ir}_{\mathclap{\text{caduta di tensione}}}
\end{equation*}
$R$: resistenza\\[\baselineskip]
Si noti che $\mathcal{E}-iR$ � l'effettiva d.d.p. che si pu� trovare ai capi di una batteria che sta
alimentando un circuito con $R>0$.

\subsubsection{Potenza elettrica}
\begin{equation*}
P = Vi
\end{equation*}
Se per conduttori ohmici
\begin{equation*}
P = iR^2 = \frac{V^2}{R}
\end{equation*}

\subsection{Effetto Joule}
L'effetto Joule � uno dei tre effetti causati da una corrente che attraversa un conduttore. In questo
caso determina il fatto che ogni conduttore attraversato da una corrente genera calore.
\begin{equation*}
Q = Ri^2t
\end{equation*}
$Q$: quantit� di calore in Joule\\
$t$: tempo

\subsection{Leggi di Kirchhoff}
Per capire queste leggi, � necessario aver chiaro tre termini
\begin{description}
	\item[Nodo] Un punto in cui convergono almeno 3 conduttori
	\item[Ramo] Una parte di circuito tra 2 nodi
	\item[Maglia] Una parte di circuito attraversata solo una volta per tornare al nodo di origine
\end{description}
Per spiegarli, prendiamo il seguente circuito
\begin{center}
	\begin{circuitikz}\ctikzset{bipoles/length=.8cm}
		\draw (0,0) 
		to[short] (1,0)
		to[R,-o](2,0)
		to[short] (3,0)
		to[battery1] (3,-1)
		to[short] (3,-2)
		to[short,-o] (2,-2)
		to[short](0,-2)
		to[battery1](0,-1)
		to[short](0,0);
		\draw (2,0)
		to[R] (2,-2);
	\end{circuitikz}
\end{center}
Vediamo due nodi (gi� evidenziati dai cerchi). Sono presenti 3 rami (tutti i modi per arrivare
da un nodo all'altro) e 3 maglie (la maglia sinistra con 2 resistenze e una batteria, quellad destra
con una sola resistenza e una batteria e quella che comprende i rami esterni).\\
Una caratteristica da ricordare � che \emph{ci sono tante correnti quanti rami}. Risolvere questi
circuiti quindi significa trovare intensit� e verso di queste correnti.

\subsubsection{Legge di Kirchhoff dei nodi}
La corrente entrante in un nodo � pari a quella uscente.
\begin{equation*}
\sum_k i_k = 0
\end{equation*}
Questo � anche definito come \textbf{Principio di conservazione di carica}.

\subsubsection{Legge di Kirchhoff delle maglie}
Per arrivare alla formula, prendiamo come esempio il seguente circuito
\begin{center}
	\begin{circuitikz}\ctikzset{bipoles/length=.8cm}
		\draw (0,0)
		to[battery1 = $\mathcal{E}_1$] (0,-1)
		to[R = $R_1$] (0,-2)
		to[R = $R_2$] (4,-2)
		to[R = $R_3$] (4,-1)
		to[battery1 = $\mathcal{E}_2$] (4,0)
		to[battery1 = $\mathcal{E}_3$] (2,0)
		to[R = $R_4$] (1,0)
		to (0,0);
	\end{circuitikz}
\end{center}
Kirchhoff descrive che
\begin{equation*}
\sum_k \mathcal{E}_k = \sum_j R_ji_j
\end{equation*}
Ovvero scegliendo un verso (orario o antiorario) se la corrente ha stesso verso le si da il valore $+$,
altrimenti $-$. Stessa cosa per la forza elettromotrice. Proseguendo da un nodo qualsiasi secondo il
verso scelto e tornando sullo stesso nodo (attraversando una maglia), sommiamo algebricamente le
forze o le resistenze che si incontrano. Si ottiene quindi la formula descritta.

\subsubsection{Utilizzo delle leggi di Kirchhoff}
Negli esercizi non possiamo usare direttamente queste leggi. Il modo generale di approcciarsi �
\begin{enumerate}
	\item Individuare le incognite (le correnti) e scegliere un verso di percorrenza in modo arbitrario
	\item Creare un sistema a $n$-equazioni dove $n$ � il numero di incognite (in generale 3)
	\item Come prima equazione, usare la LKN (Legge di Kirchhoff dei Nodi) scegliendo arbitrariamente
	i versi delle correnti nel nodo
	\item Come seconda e terza equazione, descrivere le maglie seguendo il verso scelto
	\item Risolvere il sistema lineare
\end{enumerate}