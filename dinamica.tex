%!TEX ROOT=formularioFisica.tex

\section{Dinamica}\label{sec:dinamica}
La dinamica si occupa di studiare le forze che interrcorrono tra corpi e che li fanno muovere.\\
Per gli esercizi si vada \hyperref[ex:dinamica]{qui}.
\subsection{Secondo Principio della Dinamica}
\begin{equation*}
\vec{F} = m\vec{a}
\end{equation*}

\subsection{Attrito}
Ci sono vari tipi di attrito: \emph{statico}, \emph{dinamico} e \emph{volvente} ma tutti si basano
sullo stessa idea: moltiplicare il coefficiente di attrito tra le superfici per la forza premente.
\begin{equation*}
F_d = \mu_dF_p
\end{equation*}
\begin{equation*}
F_s = \mu_sF_p
\end{equation*}
$F_p$: forza premente

\subsection{Piano Inclinato}
\begin{center}
	\begin{tikzpicture}
		\coordinate (A) at (0,0);
		\coordinate (B) at (5,0);
		\coordinate (C) at (0,2);
		\coordinate (P) at (1, 1.8); % Block center
		
		\draw[very thin] (A) -- (B) -- (C) -- cycle; % Triangle
		\draw[thin] ($(P) + (-0.2,0.3)$) -- ($(P) + (-0.4,-0.01)$) -- 
		($(P) + (0.2,-0.25)$) -- ($(P) + (0.3,0.1)$) -- cycle; % Block
		\draw[-latex, very thick, red] (P) -- ($(P) + (0, -0.5)$)
			node[pos=0.5, below right]{$\vec{P}$}; % P
		\draw[-latex, thick, blue] (P) -- ($(P) + (-0.2, -0.45)$)
			node[pos=0.5, left]{$P_\perp$}; % P_perp
		\draw[-latex, thick, cyan] (P) -- ($(P) + (0.3, -0.15)$)
			node[pos=0.5, above right]{$P_\Vert$}; % P_par
		\filldraw[orange, fill=orange!30] (B) -- ($(B) + (-1, 0)$) arc(180:158:1) -- cycle;
		\draw[orange] (B) +(170:1.2) node{$\alpha$}; % Arc
	\end{tikzpicture}
\end{center}
\begin{equation*}
\mathcolor{blue}{P_\perp} = \mathcolor{red}{P}\cos\mathcolor{orange}{\alpha}
\end{equation*}
\begin{equation*}
\mathcolor{cyan}{P_\Vert} = m\cdot a_x = \mathcolor{red}{P}\sin\mathcolor{orange}{\alpha}
\end{equation*}
\begin{equation*}
a_x = g\sin\mathcolor{orange}{\alpha}
\end{equation*}
\hyperref[tab:g]{$g$}: $9.81\,\text{m/s}^2$\\
$m$: massa del corpo\\
$a_x$: accelerazione sul piano inclinato\\
$\mathcolor{red}{\vec{P}}$: vettore della forza peso. Si noti che $\mathcolor{red}{P}$ � il suo 
modulo\\
$\mathcolor{orange}{\alpha}$: alzo del piano

\subsection{Funi e carrucole}
\begin{center}
	\begin{tikzpicture}
		\coordinate (A) at (0,0);
		\coordinate (B) at (5,0);
		\coordinate (C) at (0,2);
		\coordinate (D) at ($(C) + (-0.2, 0.2)$); % Circle center
		\coordinate (M1) at ($(D) + (-0.2, -0.5)$);
		\coordinate (M2) at ($(D) + (1, -0.25)$);
		
		\draw[very thin] (A) -- (B) -- (C) -- cycle; % Triangle
		\filldraw[orange, fill=orange!30] (B) -- ($(B) + (-1, 0)$) arc(180:158:1) -- cycle;
		\draw[orange] (B) +(170:1.2) node{$\alpha$}; % Arc
		\filldraw[fill=teal!20, very thin] (D) circle (0.2)
			node[teal]{$m$}; % Circle
		\draw[very thin] (D) -- (C); % Line from C to circle
		\draw[very thin] ($(D) + (-0.2, 0)$) -- (M1);
		\draw[red] ($(M1) + (-0.2, 0)$) -- ($(M1) + (-0.2, -0.3)$) -- ($(M1) + (0.2, -0.3)$) --
		($(M1) + (0.2, 0)$) -- cycle
			node[pos=0.5, below]{$m_1$}; % Box 1
		\draw[very thin] ($(D) + (0.15, 0.15)$) -- (M2);
		\draw[blue] ($(M2) + (0.1, 0.2)$) -- ($(M2) + (0.5, 0.01)$) -- ($(M2) + (0.35, -0.4)$) --
		($(M2) + (-0.1, -0.2)$) -- cycle
			node[pos=0, right]{$m_2$};
	\end{tikzpicture}
\end{center}
\begin{equation*}
a = \frac{\mathcolor{red}{m_1}g - \mathcolor{blue}{m_2}g\sin\mathcolor{orange}{\alpha}}
{\mathcolor{red}{m_1}+\mathcolor{blue}{m_2} + \frac{\mathcolor{teal}{m}}{2}}
\end{equation*}
Si noti che questa formula � particolare per questo caso. Si noti anche che 
$\dfrac{\mathcolor{teal}{m}}{2}$ �
da aggiungere solo se si ha la massa della carrucola e quindi il suo peso non � trascurabile.\\
Per una formula pi� generale si usi
\begin{equation*}
a= \frac{\sum \vec{F}}{\sum m}
\end{equation*}
Per gli esercizi relativi a queste tre sottosezioni, si vada \hyperref[ex:dinamica:piano]{qui}.
\label{subsec:dinamica:piano}

\subsection{Lavoro, Energia e Potenza}\label{subsec:dinamica:potenziale}
\begin{equation*}
\vec{L} = \vec{F} \cdot \vec{S} = \Delta E
\end{equation*}
\begin{equation*}
E_c = \frac{1}{2}mv^2
\end{equation*}
\begin{equation*}
P = \frac{L}{t}
\end{equation*}
\begin{equation*}
U = mgh
\end{equation*}
\begin{equation*}
U_1 + E_1 = U_2 + E_2
\end{equation*}
\hyperref[tab:g]{$g$}: $9.81\,\text{m/s}^2$\\
$L$: lavoro\\
$F$: forza\\
$S$: spostamento\\
$E_c$: energia cinetica\\
$P$: potenza\\
$U$: energia potenziale\\
$m$: massa del corpo\\
$v$: velocit� del corpo\\
\\
Il \emph{Lavoro} � un vettore che si compone della forza applicata sul corpo e del suo spostamento.\\
L'\emph{Enegia cinetica} � l'energia di un corpo in movimento.\\
La \emph{Potenza} indica la quantit� di energia scambiata nel tempo.\\
L'\emph{Energia potenziale} � quanto energia potrebbe generare quel corpo. Dipende dall'altezza.\\

Queste sono formule generali, che danno le definizioni. Negli esercizi verr� particolarmente utile
$U_1 + E_1 = U_2 + E_2$ in quanto permette di trovare la velocit� o la massa o l'altezza di un corpo
in due diverse situazioni.\\
Per gli esercizi si vada \hyperref[ex:potenziale]{qui}.\\\\
Vediamo ora alcune formule particolari:
\subsubsection{Legge di Hooke e energia elastica}
\begin{equation*}
\vec{F} = -k\vec{x}
\end{equation*}
\begin{equation*}
U_e = \frac{1}{2}k\Delta x^2
\end{equation*}
$k$: costante elastica della molla\\
$\vec{x} \approx \Delta x$: variazione di posizione\\
\\
Queste formule sono relative a molle.\\

\subsection{Quantit� di moto e teorema dell'impulso}\label{subsec:dinamica:impulso}
\label{subsec:qtaMoto}
\begin{equation*}
\vec{q} = m\vec{v}
\end{equation*}
\begin{equation*}
\vec{I} = \Delta \vec{q} = \vec{F}\Delta t
\end{equation*}
$m$: massa del corpo\\
$\vec{v}$: velocit� del corpo\\
$\vec{F}$: forza applicata sul corpo\\
$t$: tempo\\
\\
La quantit� di moto indica la forza necessaria a fermare un oggetto in movimento in un dato lasso di
tempo.\\
Per gli esercizi si vada \hyperref[ex:impulso]{qui}.

\subsection{Urti}\label{subsec:dinamica:urti}
Si distinguono 2 tipi di urti: \emph{elastici} e \emph{anaelastici}.\\
\\
Negli urti \emph{elastici} i due corpi collidono ma rimangono separati. Ad esempio due palle
da biliardo.\\
Negli urti \emph{anaelastici} i due corpi che collidono rimangono attaccati l'uno all'altro,
come nel caso di un pesce che ne mangia un alro o di un proiettile che colpisce un sacco.\\

Per gli esercizi si vada \hyperref[ex:urti]{qui}.

\subsubsection{Elastico}
\begin{equation*}
v_{1_f} = \frac{v_{1_i}\left(m_1-m_2\right) + 2m_2v_{2_i}}{m_1+m_2}
\end{equation*}
\begin{equation*}
v_{2_f} = \frac{2m_1v_{1_i} + v_{2_i}\left(m_2-m_1\right)}{m_1+m_2}
\end{equation*}
$_1$: relativo al primo corpo\\
$_2$: relativo al secondo corpo

\subsubsection{Anaelastico}
\begin{equation*}
m_1v_1 + m_2v_2 = v(m_1+m_2)
\end{equation*}
$_1$: relativo al primo corpo\\
$_2$: relativo al secondo corpo

\subsubsection{Proiettile contro un corpo}
\begin{center}
	\begin{tikzpicture}
		\coordinate (ROT) at (1, 2);
		\coordinate (CenterB1) at ($(ROT) + (0, -1.5)$);
		\coordinate (CenterB2) at ($(ROT) + (2, -1.3)$);
		\coordinate (CenterP) at (0,0.5);
		
		\draw (ROT) -- ($(CenterB1) + (0, 0.5)$);
		\draw (ROT) -- ($(CenterB2) + (-0.5, 0.5)$);
		
		\draw[very thin] ($(CenterB1) + (-0.5, -0.5)$) -- ($(CenterB1) + (-0.5, +0.5)$) --
		($(CenterB1) + (0.5, 0.5)$) -- ($(CenterB1) + (0.5, -0.5)$) -- cycle % Box1
			node[pos=0.5, above, red]{$M$};
		\draw[very thin,
		densely dashed] ($(CenterB2) + (-0.5, -0.5)$) -- ($(CenterB2) + (-0.5, +0.5)$) --
		($(CenterB2) + (0.5, 0.5)$) -- ($(CenterB2) + (0.5, -0.5)$) -- cycle;
		\draw[densely dotted] (CenterB1) -- ($(CenterB1) + (2.5, 0)$);
		\draw[blue, thick] ($(CenterB1) + (2, 0)$) -- (CenterB2)
			node[pos=0.5, right]{$h$};
		\draw ($(CenterP) + (-0.2, -0.1)$) -- ($(CenterP) + (0.2, -0.1)$) --
		($(CenterP) + (0.2, 0.1)$) -- ($(CenterP) + (-0.2, 0.1)$) -- cycle
			node[teal, below]{$\vec{v_p}$}
			node[teal, above left]{$m$};
	\end{tikzpicture}
\end{center}
\begin{equation*}
\mathcolor{teal}{v_p} = 
\frac{\mathcolor{red}{M} + \mathcolor{teal}{m}}{\mathcolor{teal}{m}}\sqrt{2g\mathcolor{blue}{h}}
\end{equation*}
\begin{equation*}
\mathcolor{blue}{h} = 
\frac{\mathcolor{teal}{v_p}^2\mathcolor{teal}{m}^2}{2g\left(\mathcolor{red}{M} +
\mathcolor{teal}{m}\right)^2}
\end{equation*}
\hyperref[tab:g]{$g$}: $9.81\,\text{m/s}^2$

\subsubsection{Urti obliqui}
Un urto obliquo � come nel seguente caso, ovvero quando due corpi collidono e partono in direzioni
diverse.\\
\begin{center}
	\begin{tikzpicture}
		\coordinate (C1) at (-1,0);
		\coordinate (C2) at (1,0);
		\coordinate (D) at (1.3,0);
		\coordinate (E1) at (2.5, 0.5);
		\coordinate (E2) at (2.5, -0.7);
		
		\draw[orange] (C1) circle (0.25)
			node[]{A};
		\draw[teal] (C2) circle (0.25)
			node[]{B};
		\draw[-latex] ($(C1) + (0.35,0)$) -- ($(C2) + (-0.35,0)$);
		\draw[-latex, dashed] (D) -- ($(D) + (1.5, 0)$);
		\draw[-latex] (D) -- (E1);
		\draw[-latex] (D) -- (E2);
		\filldraw[blue, fill=blue!30] (D) -- ($(D) + (0.4,0)$) arc(0:9:1) -- cycle;
		\draw[blue] (D) +(10:0.5) node{$\alpha$};
		\filldraw[red, fill=red!30] (D) -- ($(D) + (0.4,0)$) arc(0:-13:1) -- cycle;
		\draw[red] (D) +(-13:0.7) node{$\beta$};
	\end{tikzpicture}
\end{center}
\begin{equation*}
\vec{q}
\begin{cases*}
\mathcolor{orange}{m_A}\mathcolor{orange}{v_a}\sin\mathcolor{red}{\alpha}-
\mathcolor{teal}{m_B}\mathcolor{teal}{v_b}\sin\mathcolor{blue}{\beta} = v_0m\\
\mathcolor{orange}{m_A}\mathcolor{orange}{v_a}\cos-\mathcolor{red}{\alpha}-
\mathcolor{teal}{m_B}\mathcolor{teal}{v_b}\cos-\mathcolor{blue}{\beta} = 0
\end{cases*}
\end{equation*}

\subsection{Centro di Massa}\label{subsec:dinamica:cm}
Il centro di massa � un punto in un corpo. In quel punto si potrebbe concentrare tutta la massa del 
corpo per renderlo puntiforme.\\
Le formule riportate possono valere per tutte le dimensioni, qui per�
verr� presa in considerazione solo una per semplicit�.

\begin{alignat*}{3}
x &= \frac{\sum\limits_{i=0}^{n} m_ix_i}{\sum\limits_{i=0}^{n} m_i} &\quad
\vec{v}_{CM} &= \frac{\sum\limits_{i=0}^{n} m_i\vec{v_i}}{\sum\limits_{i=0}^{n} m_i} &\quad
\vec{a}_{CM} &= \frac{\sum\limits_{i=0}^{n} m_i\vec{a_i}}{\sum\limits_{i=0}^{n} m_i}
\end{alignat*}
Per gli esercizi si vada \hyperref[ex:cm]{qui}.

\subsection{Momento Angolare e Inerzia}\label{subsec:dinamica:inerzia}
Il momento angolare ($\vec{L}$) � la quantit� di moto per le rotazioni.\\
Al concetto di \emph{momento angoalare} si accompagna anche quello di \textbf{momento di inerzia}. 
L'inerzia ($I$) indica quanto un corpo si oppone alla rotazione.\\
\begin{equation*}
\vec{L} = \vec{r} \times \vec{q}
\end{equation*}
Da qui si nota la relazione stretta con la \nameref{subsec:qtaMoto}.
\begin{equation*}
L = rq\sin\alpha = r\cdot v\cdot m\cdot q\sin\alpha = mr^2\omega\sin\alpha = I\omega\sin\alpha
\end{equation*}
\begin{equation*}
I = mr^2 = \sum\limits_{i=0}^{n}m_ir_i^2
\end{equation*}
$m$: massa del corpo\\
$\omega$: velocit� angolare\\
$v$: velocit� tangenziale\\
$\alpha$: angolo di rotazione\\
$q$: quantit� di moto\\
\\
Se l'\textbf{asse} su cui � applicata l'inerzia non coincide con quello di rotazione ma gli \textbf{� 
parallelo},si usi questa formula
\begin{equation*}
I = I_{CM}+md^2
\end{equation*}
In aggiunta al momento angolare e al momento di inerzia, c'� la \textbf{forza angolare} ($\vec{M}$).
Molto semplicemente � definita
\begin{equation*}
\vec{M} = I\vec{\alpha}
\end{equation*}
\begin{equation*}
\Delta L = Mat
\end{equation*}
che � una forma molto simile a $\vec{F} = m\vec{a}$. Infatti $\alpha$ identifica 
\hyperref[subsec:mrua]{l'accelerazione angolare}.\\
Studiando la statica di un corpo rigido, si dimostra che
\begin{equation*}
\vec{M} = \vec{r} \times \vec{F}
\end{equation*}

\subsubsection{Teorema di K�nig}
Il teorema di K�nig descrive un moto roto-traslato. Ad esempio una ruota che si muove sull'asfalto
(ruota sul suo asse e trasla sull'asfalto).
\begin{equation*}
E_c = \frac{1}{2} I\omega^2 + \frac{1}{2}mv_{CM}^2
\end{equation*}
$I$: inerzia\\
$\omega$: velocit� angolare\\
$v_{CM}$: velocit� del centro di massa\\\\
Per gli esercizi si vada \hyperref[ex:inerzia]{qui}.