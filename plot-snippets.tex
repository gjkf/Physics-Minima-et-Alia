\usepackage{amsmath,amsfonts,amssymb,mathtools,calrsfs,mathrsfs,amsthm}

\usepackage[pdftex,dvipsnames,table]{xcolor} % Colors for text
%\usepackage{xcolor}
\usepackage{mathabx} % For better \land, \lor and \lnot
\usepackage{bm}
\usepackage{cancel}
\usepackage{siunitx}

\usepackage{tikz}
\usepackage{pgfplots}
\pgfplotsset{compat=1.14}
\usepgfplotslibrary{fillbetween}
\usepgfplotslibrary{external}
\tikzsetexternalprefix{figures/}
\tikzexternalize
\usetikzlibrary{%
  arrows,
  decorations,
  decorations.pathmorphing,
  decorations.markings,
  calc,
  intersections,
  matrix,
  spy,
  shapes.geometric,
  shapes.misc,
  patterns,
  math
}
\usepackage{tkz-euclide}
\usepackage{tikz-qtree}
\usepackage[siunitx]{circuitikz}
% http://texdoc.net/texmf-dist/doc/latex/circuitikz/circuitikzmanual.pdf
\usepackage{tikz-3dplot}

% axis style, ticks, etc
\pgfplotsset{every axis/.append style={%
    axis lines=middle,     % put the axis in the middle
    axis line style={->}, % arrows on the axis
    xlabel={$x$},          % default put x on x-axis
    ylabel={$y$},          % default put y on y-axis
    zlabel={$z$},          % default put z on z-axis
    axis equal,            % 1:1 ratio
    grid=both,             % coordinate grid
    grid style={line width=.1pt, draw=gray!10},
    major grid style={line width=.2pt,draw=gray!50},
    ticks=both,            % ticks for integers
    minor tick num=5,      % number of subticks
    ticklabel style={font=\small,fill=white},
    xlabel style={at={(ticklabel* cs:1)},anchor=north west},
    ylabel style={at={(ticklabel* cs:1)},anchor=south west},
    zlabel style={at={(ticklabel* cs:1)},anchor= west},
  }
}

\tikzset{>=stealth}
% Integral bars
\pgfplotsset{%
  integral segments/.code={\pgfmathsetmacro\integralsegments{#1}},
  integral segments=3,
  integral/.style args={#1:#2}{%
    ybar interval,
    domain=#1+((#2-#1)/\integralsegments)/2:#2+((#2-#1)/\integralsegments)/2,
    samples=\integralsegments+1,
    x filter/.code=\pgfmathparse{\pgfmathresult-((#2-#1)/\integralsegments)/2}
  }
}

\makeatletter
\def\parsenode[#1]#2\pgf@nil{%
  \tikzset{label node/.style={#1}}
  \def\nodetext{#2}
}
\tikzset{%
  % define style for the points
  Point/.style={%
    shape=circle,
    inner sep=0pt,
    minimum size=3pt,
  },
  add node at x/.style 2 args={%
    name path global=plot line,
    /pgfplots/execute at end plot visualization/.append={%
      \begingroup
        \@ifnextchar[{\parsenode}{\parsenode[]}#2\pgf@nil
        \path [name path global = position line #1-1]
        ({axis cs:#1,0}|-{rel axis cs:0,0}) --
        ({axis cs:#1,0}|-{rel axis cs:0,1});
        \path [xshift=1pt, name path global = position line #1-2]
        ({axis cs:#1,0}|-{rel axis cs:0,0}) --
        ({axis cs:#1,0}|-{rel axis cs:0,1});
        \path [
        name intersections={
          of={plot line and position line #1-1},
          name=left intersection
        },
        name intersections={
          of={plot line and position line #1-2},
          name=right intersection
        },
        label node/.append style={pos=1}
        ] (left intersection-1) -- (right intersection-1)
        node [label node]{\nodetext};
        % ---------------------------------------------------------
        % draw the tangent line from a bit right of the point on
        % the curve to the intersection with the ordinate
        % and draw the corresponding points
        %\draw [blue,thick] let
        %\p1=($ (left intersection-1) - (right intersection-1) $),
        %\p2=($ (left intersection-1)!sign(#1)*5mm!(right intersection-1) $),
        %\p3=($ ({axis cs:0,0}) - (\p2) $),
        %\n1={\p3/\p1}
        %in
        %(\p2) -- +($ {\n1}*(\x1,\y1) $)
        %%                            node [Point,fill=Tangent] (origin intersection) {}
        %%                            node [Point,fill=Curve] at (left intersection-1) {}
        %;
        %                    % ----------
        %                    % draw the horizontal line at the curve intersection point
        %                    % plus the label above/below the line
        %                    \tikzmath{
        %                        coordinate \c1;
        %                        \c1=(left intersection-1) - (right intersection-1);
        %                        \slope=\cy1/\cx1*sign(#1);
        %                    }
        %                    \pgfmathsetmacro{\AboveBelow}{ \slope>0 ? "above" : "below" }
        %                    \draw [dotted]
        %                        ([xshift=sign(#1)*2.5mm] left intersection-1) --
        %                        (left intersection-1) --
        %                            node [\AboveBelow,node font=\scriptsize] {$f(x)$}
        %                        (left intersection-1 -| origin intersection) --
        %                        +($ sign(#1)*(-2.5mm,0) $)
        %                            coordinate [pos=0.5] (a)
        %                    ;
        %                    % draw the horizontal line at the ordinate intersection point
        %                    \draw [dotted] (origin intersection)
        %                        +($ sign(#1)*(-2.5mm,0) $) --
        %                        (origin intersection);
        %                    % draw vertical line left/right of the ordinate
        %                    \pgfmathsetmacro{\LeftRight}{ #1<0 ? "right" : "left" }
        %                    \draw [stealth-stealth] (origin intersection)
        %                        +($ sign(#1)*(-1.25mm,0) $) -- (a)
        %                            node [midway,\LeftRight,node font=\scriptsize] {$p$}
        %                    ;
        %                    % ---------------------------------------------------------
      \endgroup
    },
  },
}
\makeatother

\DeclarePairedDelimiter\norm{\lVert}{\rVert} % ||v||
\DeclarePairedDelimiter\abs{\lvert}{\rvert} % |v|

\makeatletter
\def\mathcolor#1#{\@mathcolor{#1}}
\def\@mathcolor#1#2#3{%
  \protect\leavevmode
  \begingroup
  \color#1{#2}#3%
  \endgroup
}
\makeatother
\newcommand{\Setsuchthat}{\;\ifnum\currentgrouptype=16 \middle\fi|\;}
\newcommand{\suchthat}{\,:}

\newcommand*\dif{\mathop{}\!\mathrm{d}}
\newcommand*\Dif{\mathop{}\!\mathrm{D}}
\newcommand*{\der}[2]{\frac{\dif #1}{\dif #2}}

\DeclareMathOperator{\arcsec}{arcsec}
\DeclareMathOperator{\arccot}{arccot}
\DeclareMathOperator{\arccsc}{arccsc}

% For an older but clearer root. Still \oldsqrt is valid
\usepackage{letltxmacro}
\makeatletter
\let\oldr@@t\r@@t
\def\r@@t#1#2{%
  \setbox0=\hbox{$\oldr@@t#1{#2\,}$}\dimen0=\ht0
  \advance\dimen0-0.2\ht0
  \setbox2=\hbox{\vrule height\ht0 depth -\dimen0}%
{\box0\lower0.4pt\box2}}
\LetLtxMacro{\oldsqrt}{\sqrt}
\renewcommand*{\sqrt}[2][\ ]{\oldsqrt[#1]{#2} }
\makeatother

\newcommand\markangle[7][red]{% [color] origin A B radius radiusmark mark
  % fill red circle
  \begin{scope}
    \path[clip] (#2) -- (#3) -- (#4);
    \fill[color=#1,fill opacity=0.5,draw=#1,name path=circle]
    (#2) circle (#5);
  \end{scope}
  % middle calculation
  \path[name path=line one] (#2) -- (#3);
  \path[name path=line two] (#2) -- (#4);
  \path[%
  name intersections={of=line one and circle, by={inter one}},
  name intersections={of=line two and circle, by={inter two}}
  ] (inter one) -- (inter two) coordinate[pos=.5] (middle);
  % put mark
  \node at ($(#2)!#6!(middle)$) {#7};
}

\newcommand\domainplot[3]{ % start end list \domainplot{0}{2}{0,1}
  \draw[shorten <= -1cm,shorten >= -1cm] (#1) -- (#2);
  \foreach \x in #3{
    \coordinate (A) at (#1);
    \coordinate (B) at (\x,0);
    \filldraw[fill=white,draw=black] (A-|B) circle(2pt)
    node[below]{\x};
  }
}

\newcommand{\mysign}[1]{%
  \pgfmathtruncatemacro\tmpsign{sign(#1)}
  \ifnum\tmpsign<0
  -
  \else
  \ifnum\tmpsign>0
  +
  \else
  \relax
  \fi
  \fi
}

\newcommand{\mytick}[1]{%
  \pgfmathtruncatemacro\tmpsign{sign(#1)}
  \ifnum\tmpsign<0
  \relax
  \else
  \ifnum\tmpsign>0
  \relax
  \else
  $|$
  \fi
  \fi
}

\newcommand{\drawSign}[6][]{% f(x) xmin xmax delta exclpoints
  \begin{scope}[#1,declare function={Test(\x) = #2;}]
    \draw [->, thick] (#3,0) -- (#4,0)node[right] {$x$};
    \pgfmathsetmacro{\NextI}{#3+#5}
    \foreach \i in {#3,\NextI,...,#4}{%
      \node at (\i,{0.35*sign(max(min(Test(\i),1),-1))}) {$\mysign{Test(\i)}$};
      \pgfmathtruncatemacro\signum{sign(Test(\i))}
      \ifnum\signum=0
      \filldraw[black] (\i,0) circle(2pt);
      %\node[circle,fill,maximum size=2pt] at (\i,0){};
      \node[below] at (\i,-0.2){$\i$};
      \fi
    }
    \ifx&#6&%
    \else  
    % Excluded points
    \foreach \x in #6{%
      \filldraw[fill=white,draw=black] (\x,0) circle(2pt);
      \node[below] at (\x,-0.2) {$\x$};
    }
    \fi
  \end{scope}
}

% Arc over text
\usepackage{graphicx}
\makeatletter
\DeclareFontFamily{U}{tipa}{}
\DeclareFontShape{U}{tipa}{m}{n}{<->tipa10}{}
\newcommand{\arc@char}{{\usefont{U}{tipa}{m}{n}\symbol{62}}}%

\newcommand{\arc}[1]{\mathpalette\arc@arc{#1}}

\newcommand{\arc@arc}[2]{%
  \sbox0{$\m@th#1#2$}%
  \vbox{%
    \hbox{\resizebox{\wd0}{\height}{\arc@char}}
    \nointerlineskip
    \box0
  }%
}
\makeatother

% Matrix spacing
\makeatletter
\renewcommand*\env@matrix[1][\arraystretch]{%
  \edef\arraystretch{#1}%
  \hskip -\arraycolsep
  \let\@ifnextchar\new@ifnextchar
\array{*\c@MaxMatrixCols c}}
\makeatother

\everymath{\displaystyle}
\newcommand\ddfrac[2]{\frac{\displaystyle #1}{\displaystyle #2}}

% Proof changes
\renewcommand\qedsymbol{QED}
%\renewenvironment{proof}[1][\proofname]{{\bfseries #1.}}{}

\expandafter\let\expandafter\oldproof\csname\string\proof\endcsname
\let\oldendproof\endproof
\renewenvironment{proof}[1][\proofname]{%
  \oldproof[\ttfamily \scshape \large #1]\mbox{}\\*%
}{\oldendproof}

% Hôpital's equivalence
\newcommand{\Heq}[1]{\overset{\left[#1\right]}{\underset{\mathrm{H}}{=}}}
